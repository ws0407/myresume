%%
%% Copyright (c) 2018-2019 Weitian LI <wt@liwt.net>
%% CC BY 4.0 License
%%
%% Created: 2018-04-11
%%

% Chinese version
\documentclass[zh]{resume}
% Adjust icon size (default: same size as the text)
\iconsize{\Large}

% File information shown at the footer of the last page
\fileinfo{%
  % \faCopyright{} 2018--2020, Weitian LI \hspace{0.5em}
  \githublink{ws0407}{myresume} \hspace{0.5em}
  \creativecommons{by}{4.0} \hspace{0.5em}
  \faEdit{} \today
  \hfill 1/1
}

\name{硕}{王}

% \keywords{BSD, Linux, Programming, Python, C, Shell, DevOps, SysAdmin}

% \tagline{\icon{\faBinoculars}} <position-to-look-for>}
% \tagline{<current-position>}

% Supported shapes: circular (default), square, rectangle
% \photo[<shape>]{<width>}{<filename>}
\photo[rectangle]{1.5cm}{photo/photo.jpg}

\profile{
  \info{男 | 2000.05 | 中共党员}\hspace{1em}
  \university{清华大学}\hspace{1em}
  \degree{计算机技术 \textbullet 硕士}\hspace{1em}
  % \birthday{2000-05-10}
  \address{安徽省宿州市} \\
  \mobile{198-0123-1323}\hspace{.77em}
  \email{wangs22@mails.tsinghua.edu.cn}\hspace{.77em}
  \github{ws0407}
  % Custom information:
  % \icontext{<icon>}{<text>}
  % \iconlink{<icon>}{<link>}{<text>}
}

\begin{document}
\makeheader

% %======================================================================
% % Summary & Objectives
% %======================================================================
% {\onehalfspacing\hspace{2em}%
% 计算机技术专业(网络空间安全方向)硕士研究生,主要研究基于深度学习的网络流量识别、源代码审计等技术。
% 熟练使用C/C++、Python编程,熟悉机器学习模型、数据处理算法理论及实现。
% 热衷计算机和网络安全技术,熟练使用常见的开发、测试软件,熟悉反汇编、漏洞挖掘、Linux安全工具。
% \par}
% \vspace{-.5em}
%======================================================================
\sectionTitle{教育背景}{\faGraduationCap}
%======================================================================
\begin{educations}
  \education%
    {至今}%
    [2022.09]%
    {清华大学(研究生,保送)}%
    {计算机技术 \textbullet\ GPA:3.85/4.00}%
    {}{主要荣誉:发表学术论文一篇(CCF-A类期刊)、清华大学社会工作奖学金}

  \separator{0.1ex}
  \education%
    {2022.06}%
    [2018.09]%
    {北京邮电大学(本科)}%
    {网络空间安全(实验班) \textbullet\ GPA:3.77/4.00,排名:1/105}%
    {}{主要荣誉:本科生特等奖学金、国家奖学金、北京市三好学生、北京市优秀毕业生}

  % \separator{0.1ex}
  % \course%
  %   {主修}%
  %   [课程]%
  %   {哈哈哈1}
  %   {哈哈哈2}
\end{educations}


%======================================================================
\sectionTitle{专业技能}{\faCogs}
%======================================================================
% \Huge, \huge, \Large, \large, \small, \tiny
\begin{competences}
  \comptence{编程语言}{%
  \smallicon{\mfPython}Python, \hspace{.6em} \smallicon{\textbf{\sffamily{C\kern-.1em\raisebox{.3ex}{/}\kern-.2em C\kern-.2em \raisebox{.45ex}{\tiny{++}}}}}\hspace{.6em}, 
  \smallicon{\mfJavaBold}\hspace{-.3em}Java, 
  \smallicon{\mfHtmlfiveAlt}\hspace{-.3em}
  \smallicon{\mfCssthreeAlt}\hspace{-.3em}
  \smallicon{\mfJavascriptAlt}, 
  \smallicon{\mfPhpAlt}PHP, 
  \hspace{.6 em}\smallicon{\LaTeX}\hspace{.6 em}, 
  \smallicon{\faMarkdown}Markdown
  }
  \comptence{专业工具}{%
    SSH, 
    \smallicon{\mfGit}Git; 
    \smallicon{\mfDocker}Docker, 
    \smallicon{\faLinux}Linux, Make; 
    Burpsuite, C-Spider, SecVas, SecureCAT, SenInfo, IDA
  }
  \comptence{数据分析}{%
    PyTorch, Tensorflow;Numpy, Pandas, Matplotlib;Keras, Scikit-learn
  }
  \comptence{网站开发}{%
    Django, Flask; \smallicon{\mfJavascriptAlt}JavaScript, \smallicon{\mfJquery}jQuery, \smallicon{\mfBootstrap}Bootstrap; \smallicon{\mfMssql}\hspace{-.4 em}\smallicon{\mfMysql}SQL
  }
  \comptence{数理基础}{
    全国大学生数学竞赛\textbf{一等奖}、物理竞赛\textbf{一等奖}、美国大学生数学建模竞赛\textbf{H奖}
  }
  \comptence{外语水平}{
    \textbf{英语} --- CET4/6,学术英语课程(A),读写(优良),听说(日常交流)
  }
\end{competences}

% %======================================================================
% \sectionTitle{计算机技能}{\faCogs}
% %======================================================================
% \begin{itemize}
%   \item DragonFly BSD 操作系统开发者:
%     200+ 代码提交;内核以及系统工具;
%     在邮件列表和 IRC 频道交流和回答问题
%   \item 使用 Ansible 管理 VPS,部署个人域名邮箱、权威 DNS、网站、Git、IRC 等服务
%   \item 搭建并管理课题组的工作站、计算集群(4 节点)和网络设备
%   \item 参与配置和测试上海天文台的 SKA 高性能计算集群原型机
%     (1 管理节点 + 1 存储节点 + 4 计算节点)
%   \item 设计并开发了\enquote{2014 第一届中国—新西兰联合 SKA 暑期学校}的整个网站
%     (Django, Bootstrap, jQuery)
% \end{itemize}

%======================================================================
\sectionTitle{科研成果}{\faCode}
%======================================================================
{\color{accentcolor}{\smallicon{\faLocationArrow} \textbf{学术论文}}}
\vspace{-0.2cm}
\begin{itemize}
  \item \link{https://github.com/ws0407/RBLJAN}{\texttt{RBLJAN}}:
    提出了一个高效的深度学习框架:\textbf{高鲁棒性的字节-标签联合注意力网络},并专注于\textbf{加密网络流量分类}。该框架结合分类器与对抗流量生成器,通过并行处理和注意力学习,捕捉字节与标签之间的关联,相比于现有方法,RBLJAN具有优越的检测\textbf{准确性}、\textbf{速度}、\textbf{鲁棒性}和\textbf{可解释性}。\textbf{论文}: Robust Byte-Label Joint Attention Network for Network Traffic Classification(\textbf{TDSC}, \textbf{CCF-A}类期刊)
  \item \link{https://github.com/ws0407/code_audit}{\texttt{CodeVD}}:
    \textbf{基于大模型微调的源代码漏洞定位和解释工作研究},微调大模型学习现有数据集(提交信息、源代码、漏洞标签)和各类漏洞的先验知识(可能触发漏洞的语句PoI),从而完成更细粒度的漏洞样本构建和源代码漏洞挖掘。该技术\textbf{已在华为技术有限公司实习期间验证了有效性},论文正在撰写和投稿
\end{itemize}
\vspace{-0.1cm}
{\color{accentcolor}{\smallicon{\faGithub} \textbf{个人项目(部分)}}}
\vspace{-0.2cm}
\begin{itemize}
  \item \link{https://github.com/ws0407/NetworkAnalysisPrototypeSystem}{\texttt{NAPS}}:
    国家重点研发计划课题的\textbf{原型系统}(集成8个大型科研项目)
  \item \link{https://github.com/ws0407/BotnetAD}{\texttt{BotnetAD}}:
    基于机器学习和 DNS 流量特征分析的\textbf{僵尸网络检测}工具(Python, 随机森林, 特征挖掘)
  \item \link{https://github.com/ws0407/WVSS}{\texttt{WVSS}}:
   \textbf{ Web 漏洞扫描}系统(支持 SQL 注入、XSS、文件上传、弱口令四类漏洞的基本扫描)
  \item \link{https://github.com/ws0407/CloudVS}{\texttt{CloudVS}}:
    企业云环境的搭建及云环境下 KVM \textbf{虚拟机镜像漏洞扫描}工具 
  \item \link{https://github.com/ws0407/TrojanControlSystem}{\texttt{TrojanCS}}:
    Windows10 操作系统\textbf{远程木马控制系统}的设计与实现(客户端、服务端)
  \item \link{https://github.com/ws0407/2ndSQLDetection}{\texttt{2ndSQLD}}:
    基于源代码分析和自动化执行的\textbf{二阶 SQL 注入漏洞检测}工具 
  \item \link{https://github.com/xd0419/webpan}{\texttt{NetDisk}}:
    基于 Java SSM 和 RSA 密码体制的\textbf{安全网盘系统} 
  \item \link{https://github.com/ws0407/TEnDeDriver}{\texttt{TEnDeDriver}}:
    Win10 内核文件透明加解密驱动开发和 SIEM 日志分析框架的构建 
\end{itemize}

%======================================================================
\sectionTitle{实习/社会工作}{\faBriefcase}
%======================================================================
\begin{experiences}
  \experience%
    [2024.07]%
    {2023.07}%
    {\textbf{测试技术员}(实习) @ 华为技术有限公司 \textbullet 华为云计算(东莞)}%
    [\begin{itemize}
      \item 研究\textbf{基于大模型的源代码漏洞检测}技术,清华-华为道元班联合培养项目
      \item \textbf{JavaWeb漏洞挖掘},参与\textbf{华为云服务项目测试}(CodeArtscheck、AstroFlow等)
    \end{itemize}]
    [Java, Web渗透, 代码审计, 大模型]

  \separator{0.5ex}
  \experience%
    [2022.07]%
    {2021.07}%
    {\textbf{原型系统开发}(实习) @ 清华大学 \textbullet 网络研究院}%
    [\begin{itemize}
      \item 前后端\textbf{全栈开发},集成国家重点研发计划(互联网基础行为基准建模与异常分析)的8个子科研项目,实现多线程数据管理、任务执行和监控、结果展示与分析
    \end{itemize}]
    [Python, Django, Docker, jQuery, AJAX]
    
  \separator{0.5ex}
  \experience%
    []%
    {\hspace{1em}学生工作个人发展}%
    {清华大学\textit{院研究生会生活部\textbf{部长}}、马拉松协会成员(半马135,全马330);北京邮电大学院学生会\textbf{主席}、北邮“数学人”社团\textbf{副社长}。积极参与社会实践,志愿时长丰厚(北京690h、深圳80h)}%
    []
    []
\end{experiences}

\end{document}